乱-Ran-

Rebellion

sung from sukuna's perspective, princess of loyal descendant who is tricked and deceived by seija kijin to use the magical mallet for her own benefit
```
Original game: 东方辉针城 ~ Double Dealing Character

Original title: 空中に沈む輝針城
Theme: Stage 5 theme

Original title: 輝く針の小人族 ~ Little Princess
Theme: Stage 6 BOSS - Sukuna Shinmyoumaru's (少名 針妙丸) theme
```

何も知らずいられた 真実をここに聞くまでは \\
```
なにも: anything
しる: know (verb)
知らず:  without knowing
いられた: potential/receptive helper verb
unable to know/how can you know this (the ignorant)

                しんじつ: truth (noun)
                  ここ: here
                        こく: hear (verb)
                            まで: until (time and abstract location context)
(I did not know about anything) + (until (I) hear about it here) + (the truth)
```
The truth unbeknownst to me until I heard about it here

どれだけの苦しみが 私たちを虐げたのか \\
```
どれ: which one
だけ: only/as much as (limitation context)
          くるしみ: pain/suffering (noun)

                    わたしたち: we/us (pronoun)
                            しいたげる: oppress (ichidan verb)
                            虐げた: past tense ichidan verb
                              か: wondering context/self questioning
(just how much pain/suffering) + (the oppression done to us would be/our oppression)
```
was just how much pain and suffer our oppression caused {1}

さあ、目覚めよ 言葉持たぬ 囚われの意思 \\
```
さあ: now (callout expression)
めざめる: wake up (phrasal verb)

                言葉 (ことば): language/words (noun)
                持つ (もつ): hold/possess (verb)
                持たぬ: negative verb (old japanese)

                            とらわれ: captive/captured (noun)
                                    いし: mind/purpose (noun)
(now, wake up) (without language/speech) (mind of prison)
```
Now, wake up; Don't utter a word; Imprisoned minds

求めれば 与えよう 力を命を 憎しみ故に \\
```
もとめる: wish/request (ichidan verb)
求めれば: conditional/if/request context

          あたえる: give/grant/bestow (ichidan verb)
          与えよう: desire/questioning desire/"let's" context

                    ちから: strength/power (noun)
                    いのち: life (noun)

                              にくしみ: hatred
                              故 (ゆえ): reason/cause (noun)
(if I wish for it) ((please) grant me it) (power and life) (also hatred)
```
If you wish for it, I'll grant them to you: Power and life, also the fuel to resentment

空、翳り \\
```
そら: sky (noun)
翳 (かげ): shadow
翳り: shadow/gloom/shade (noun)
the glooming sky
```
The glooming sky

輝く針の剣を抜く \\
```
かがやく: shine/sparkle (verb)
はり: needle/pin (noun)
剣 (つるぎ): sword (noun)
ぬく: pull out/carry through (verb)
unsheathe your shining needle
```
Unsheathe your shining needle {2}

憂い胸に \\
```
うれい: sorrow/grief (noun)
むね: chest (noun)
```
The chest filled with sorrow

強き者負かすのは さらに強大な力のみ \\
```
つよい: strong (i-adjective)
強き: adverb transformation き
もの: person (noun)
まかす: defeat (smth/someone) (verb)

                さら: furthermore/even more (adverb)
                    きょうだい: strong/powerful (noun)
                    な: simile/metaphor particle
                    ちから: power (noun)
                    のみ: "only" context - same as だけ/limitation particle
(the strong who defeats) + (even bigger force/power)
```
The strong who defeats can only be even stronger

弱き者の望みは やがてその身をも滅ぼす \\
```
よわい: weak (i-adjective)
もの: person (noun)
のぞみ: wish/desire (noun)

                やがて: eventually
                その: that (thing)
                み: body (noun)
                ほろぼす: destroy (verb)
(the weak who wishes) + (will eventually destroy that body of theirs)
```
The weak who begs will eventually destroy those bodies of theirs

さあ、この世に この地上に 夢に見たような \\
```
さあ: now
          せ: world (noun)
                  ちじょう: surface/land (noun)

                            ゆめ: dream (noun)
                            見る: see (ichidan verb)
                            見た: saw (past tense ichidan verb)
                            ような: as if/like (hypothetical metaphor)
(now) (this world) (this land) (like something you'd see in a dream)
```
Now, this world, this land, just like something you'd see in a dream

楽園を 楽園を もう誰一人悲しまないように \\
```
らくえん: paradise (noun)

              もう: soon/already (adverb)
                だれ: who (pronoun)
                ひとり: one's self (noun)
                          かなしむ: be sad for (obj) (verb)
                          悲しまない: negative tense verb
                                    ように: as if/like (hypothetical metaphor)
(a paradise) (like no one would be sad)
```
A paradise, a paradise, where no one would be sad

空、翳り
輝く針の剣を抜く
痛み胸に 憂世の乱 輝針の乱
```
いたみ: pain (noun)
    むね: chest (noun)

          憂世 (ゆうせい): sad world/present world/fleeting life (noun)
             らん: rebellion (noun)
                  輝針 (きしん): shining needle
(this hurting chest) (this pitiful world of rebellion) (this shining needle of rebellion)
```
The chest filled with pain, this pitiful world of rebellion, this shining needle of rebellion {3}

やがて、 \\
```
Eventually
```

夜は明ける 朝が息吹く その手にある鬼の呪物 \\
```
よる: night (noun)
あける: grow/emit light (ichidan verb)

あさ: morning (noun)
いぶく: breathe (verb)

その: that
て: hand (noun)
ある: is there/exists (verb)
おに: (oni context) (noun)
じゅぶつ: fetish (not in sexual context but in spiritual - temple's talisman or charm. E.g ema, kumate, maneki-neko) (noun)
(light up the night) + (breathe in the morning (air)) + (with demon(referring to amanojaku) charm in hand)
```
Brighten up the night, breathe in the sunlight, take the demon charm {4}

時は来たり 潮満ちたり 力よさあ、この身を焼け!
```
とき: time (noun)
くる: come/go (verb)
来たり: coming (noun)

            潮 (しお): tide (noun)
            満ちる (みちる): full/rise(tide/water context) (ichidan verb)
            潮満ちたり (しおみつ): rising tide

                        ちきら: power/strength (noun)
                        よ + さあ: colloquial <?>
                                  この: this
                                      み: body (noun)
                                          やける: burn/get roastet (ichidan verb)
                                          やけ: commanding verb

(the time has come) (the rising tide) (with force/power, set this body on fire)
```
The coming hours, the rising tide, with force and power, set this body on fire

切り裂かれた 涙枯れた 弱き者の嘆きを聞け \\
```
きる: cut (verb)
切り: verb compound connector り
さく: tear/rip (verb)
裂かれた: past tense receptive helper verb

              なみだ: tear (noun)
              かれる: dry up (verb)
              枯れた: died up (past tense verb)

                        よわい: weak (i-adjective)
                        弱き: weakly (adverb)
                            もの: body (noun)
                                なげき: grief/lament (noun)
                                      きける: tell (old japanese) (ichidan verb)
                                      聞け: tell (commanding verb)
(cut and tear (past)) (the dried up tears) (the weak bodies tells the stories of the grief)
```
Getting cut and torn, dried up streams of tears, the weak's lamenting

語る声も ふるう拳も 死ぬ自由もないモノたち \\
```
かたる: talk/tell a story (verb)
    こえ: voice (noun)
          ふるう: swing (verb)
                こぶし: fist (noun)
                も: as well as (double も structure)
                      死ぬ(しぬ): die (verb)
                      じゆう: freedom (noun) (自ら (みずから): self (noun) + 由 (よし): reason (verb))
                              もない: even + negative
                                    モノたち: things/set of things/objects (noun)
(monotone/chanting voice) (as well as the swinging fists) (things that don't have the freedom to die/can't even die)
```
The unchanging voice, the flurry of blows, as well as beings that can't even die

私は今 全てを知り敵を討たん \\
```
わたし: I (feminine) (pronoun)
    いま: now

        すべて: everything (noun)
              しる: know (verb)
              しり: clause connector り
                てき: enemy/opponent
                      討つ (うつ): attack/shoot (verb)
                      討たん: past tense verb
(I now know everything) + (and bring vengeance to the enemies)
```
I now know everything, I'll now bring vengeance to the enemies

私は今 己を知り空に沈む \\
```
        おのれ: myself (this self) (literature pronoun)
            しる: know (verb)
            しり: adverb transformation り
                そら: sky (noun)
                    しずむ: sink/go under (verb)
(I now know myself) + (and witness the horizon)
```
I now know myself, to understand this self as the horizon falls

まだ足りない まだ足りない \\
```
だま: still/yet
たりない: negative + enough
```
It's still not enough, it's still not enough

運命にさえ抗う力を
```
うんめい: fate/destiny (noun)
さえ: even if
抗う (あらが): oppose/object/go against (verb)
ちから: force/strength (noun)
(against fate) (even if opposing force)
```
夜は明けて 朝は息吹く
```
よる: night (noun)
あける: grow/emit light (ichidan verb)
明けて: clause connector て

            あさ: morning (noun)
            いぶく: breathe (verb)
[the oni that](light up the night) + (and the breathe in the morning (air)) + (next line oni)
```
The lit up nightfall, the dawning morning

鬼が人知れず笑って
```
おに: oni context (noun)
ひとしれず: secret/unseen (人: person/man + 知れず: (negative ず) know)
わらう: laugh (verb)
笑って: commanding (can be inferred as "laughing")
(laughs in secret)
```
made by the demon that laughs in secrecy

{1} possibly seija guilt-tripping

{2} sukuna's weapon of choice, she's an inchling so a needle is actually around the length of a sword/sabre to a normal human

{3} sukuna's whereabouts, also where the magical mallet is located

{4} the song from this point onwards is sung in segments of 6 syllables so I tried to also structure the part from here on to be rhythmic, this case being 5
