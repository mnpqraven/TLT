%!TEX TS-program = xelatex
% !TeX program = xelatex
\documentclass{article}
\usepackage[margin=1in]{geometry}
\usepackage{xeCJK}
\usepackage{ruby}
\setCJKmainfont{NotoSansJP-Light.otf}
\usepackage{tikz}
\usepackage{comment}
\usepackage{indentfirst}
\usepackage[linktoc=all]{hyperref}
% characters with big height and depth
% to give different boxes the same vertical size
\newcommand{\vsizecorrectorhira}{\vphantom{もりぼゃ}}% epkyouka, HanaMinA
\newcommand{\vsizecorrectorkanji}{\vphantom{$\vert$}}

% commands for setting furigana below kanji
\newcommand{\furiganaBelow}[2]{% #1: kanji, #2: furigana
    %\unskip
    \begin{tikzpicture}[baseline=(kanji.base)]
        \node(kanji)[
            inner sep=0,
        ] {
            \vsizecorrectorkanji%
            #1%
        };
        \node(furigana)[
            below of=kanji,
            node distance=1em-2pt,
            inner sep=0,
        ] {%
            \tiny%
            \vsizecorrectorhira%
            #2%
        };
    \end{tikzpicture}%
    %\ignorespaces
}

\newcommand{\furiganaAboveBelow}[3]{% #1: kanji, #2: furigana above, #3: furigana below
    %\unskip
    \begin{tikzpicture}[baseline=(kanji.base)]
        \node(kanji)[
            inner sep=0,
        ] {
            \vsizecorrectorkanji%
            #1%
        };
        \node(furigana-above)[
            above of=kanji,
            node distance=1em,
            inner sep=0,
        ] {%
            \tiny%
            \vsizecorrectorhira%
            #2%
        };
        \node(furigana-below)[
            below of=kanji,
            node distance=1em-2pt,
            inner sep=0,
        ] {%
            \tiny%
            \vsizecorrectorhira%
            #3%
        };
    \end{tikzpicture}%
    %\ignorespaces
}

\newcommand{\furigana}[2]{% #1: kanji, #2: furigana
    %\unskip
    \begin{tikzpicture}[baseline=(kanji.base)]
        \node(kanji)[
            inner sep=0,
        ] {
            \vsizecorrectorkanji%
            #1%
        };
        \node(furigana)[
            above of=kanji,
            node distance=1em-0pt,
            inner sep=0,
        ] {%
            \scriptsize%
            \vsizecorrectorhira%
            #2%
        };
    \end{tikzpicture}%
    %\ignorespaces
}

\setlength{\parskip}{0.4em}

\title{天磐船ヨ天ヘ昇レ \\ Heavenly boat, climbimg towards the high heavens}
\author{Original Source: 暁records --- Translated by: Othi}

\begin{document}

\maketitle

\tableofcontents
\section{Origin}
\noindent Original game: 東方神霊廟 ~ Ten Desires

\noindent Original title: 大神神話伝 ~ Omiwa Legend \\
Theme: Stage 5 Boss --- Mononobe no Futo's (物部 布都) theme

\noindent Original title: 夢殿大祀廟 ~ The Hall of Dreams' Great Mausoleum \\
Theme: Stage 5 Theme

\section{Summary}
Title background

① 空中を飛行する堅固な船。「日本書紀」では、高天原から下界に降りる際に用いた船として伝えている。 \\
※書紀(720)神武天皇即位前甲寅年「天磐船(あまのいはふね)に乗りて飛び降る者有りといひき」 \\
※日本紀竟宴和歌‐延喜六年(906)「そらみつに阿麻能伊婆布然(アマノイハフネ)くだししはひじりの御代を渡すとてなり〈藤原忠紀〉」 \\
② 天の川にあるという想像上の船。 \\
※堀河百首(1105‐06頃)秋「彦星のあまの岩ふねふなでして今夜(こよひ)や磯にいそ枕する〈藤原顕仲〉」

Source: \href{https://kotobank.jp/word/%E5%A4%A9%E3%81%AE%E7%A3%90%E8%88%B9-198533}{Link}

\pagebreak

\section{Lyric}
\subsection{Intro}
\subsection{Verse \#1}

どれくらい\furigana{闇}{やみ}は\furigana{続}{つづ}く? \\
How far will you continue in the darkness?

\furigana{最後}{さいご}に\furigana{触}{ふ}れたあなたの\furigana{指先}{ゆびさき} \\
Even just the heat

\furigana{温度}{おんど}さえも \\
 of your fingertips' final touch

 どれくらい\furigana{落}{お}ちたらいい? \\
Just how deep could you have fallen down?

\furigana{流星}{りゅうせい}\furigana{宙}{そら}に\furigana{燃}{も}える \\
Burning meteors in the sky

\furigana{優}{やさ}しく\furigana{手}{て}にかけて \\
By your gentle hands

また\furigana{愛}{あい}し\furigana{合}{あ}うために \\
I'll love you all over again for your sake

\subsection{Chorus \#1}

\furigana{天磐船}{あまのいわふね}よ \\
It's the heavenly boat

\furigana{天}{てん}ヘ\furigana{天}{てん}ヘノボレ \\
Climb towards the heavens

\furigana{再}{ふたた}び\furigana{空}{そら}を\furigana{割}{わ}いて \\
Cleave the sky once more and

\furigana{天}{てん}ヘ\furigana{天}{てん}ヘタカク \\
Towards the high heavens

\subsection{Verse \#2}

\furigana{示}{しめ}せる\furigana{愛}{あい}があるなら \\
If I can show that this love is real

それが\furigana{死}{し}だとしても \\
Even if this would be death

\furigana{差}{さ}し\furigana{出}{だ}してみせましょう \\
I'd like to submit them to you

この\furigana{身}{み}を\furigana{魂}{たましい}を \\
This body and soul of mine

\subsection{Chorus \#2}

\furigana{天磐船}{あまのいわふね}よ \\
It's the heavenly boat

\furigana{天}{てん}ヘ\furigana{天}{てん}ヘノボレ \\
Climb towards the heavens

\furigana{優}{やさ}しくその\furigana{手}{て}で\furigana{殺}{あや}めて \\
End me with those gentle hands of yours

そして、 \\
at last.

\subsection{Outro}

\furigana{現世}{うつしよ}はまた\furigana{神}{かみ}を\furigana{喪}{うしな}うだろう \\
Suppose we lose our god in this age once more

その\furigana{眠}{ねむ}り\furigana{覚}{き}めることないままに \\
Then they would not wake up from their slumber ever again

\furigana{天翔}{あまか}ける\furigana{磐船}{いわふね}よ \furigana{天}{てん}ヘノボレ \\
The soaring heavenly boat, climbs towards the heavens

\furigana{息絶}{いきた}えた\furigana{神}{かみ}のせて \\
Let the gods be laid to rest

どこまでも、どこまでも \\
Anywhere, everywhere

\furigana{闇}{やみ}の\furigana{中}{なか}。 \\
In the pitch of darkness.
\end{document}
