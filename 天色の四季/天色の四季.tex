%!TEX TS-program = xelatex
% !TeX program = xelatex
\documentclass{article}
\usepackage[margin=1in]{geometry}
\usepackage{xeCJK}
\usepackage{ruby}
\setCJKmainfont{NotoSansJP-Light.otf}
\usepackage{tikz}
\usepackage{comment}
\usepackage{indentfirst}
\usepackage[linktoc=all]{hyperref}
% characters with big height and depth
% to give different boxes the same vertical size
\newcommand{\vsizecorrectorhira}{\vphantom{もりぼゃ}}% epkyouka, HanaMinA
\newcommand{\vsizecorrectorkanji}{\vphantom{$\vert$}}

% commands for setting furigana below kanji
\newcommand{\furiganaBelow}[2]{% #1: kanji, #2: furigana
    %\unskip
    \begin{tikzpicture}[baseline=(kanji.base)]
        \node(kanji)[
            inner sep=0,
        ] {
            \vsizecorrectorkanji%
            #1%
        };
        \node(furigana)[
            below of=kanji,
            node distance=1em-2pt,
            inner sep=0,
        ] {%
            \tiny%
            \vsizecorrectorhira%
            #2%
        };
    \end{tikzpicture}%
    %\ignorespaces
}

\newcommand{\furiganaAboveBelow}[3]{% #1: kanji, #2: furigana above, #3: furigana below
    %\unskip
    \begin{tikzpicture}[baseline=(kanji.base)]
        \node(kanji)[
            inner sep=0,
        ] {
            \vsizecorrectorkanji%
            #1%
        };
        \node(furigana-above)[
            above of=kanji,
            node distance=1em,
            inner sep=0,
        ] {%
            \tiny%
            \vsizecorrectorhira%
            #2%
        };
        \node(furigana-below)[
            below of=kanji,
            node distance=1em-2pt,
            inner sep=0,
        ] {%
            \tiny%
            \vsizecorrectorhira%
            #3%
        };
    \end{tikzpicture}%
    %\ignorespaces
}

\newcommand{\furigana}[2]{% #1: kanji, #2: furigana
    %\unskip
    \begin{tikzpicture}[baseline=(kanji.base)]
        \node(kanji)[
            inner sep=0,
        ] {
            \vsizecorrectorkanji%
            #1%
        };
        \node(furigana)[
            above of=kanji,
            node distance=1em-0pt,
            inner sep=0,
        ] {%
            \scriptsize%
            \vsizecorrectorhira%
            #2%
        };
    \end{tikzpicture}%
    %\ignorespaces
}

\newcommand{\tlnotelink}[1]{\hyperlink{#1tonote}{[#1]}\hypertarget{#1}{}}
\newcommand{\tlnoteref}[1]{\hypertarget{#1tonote}{}\hyperlink{#1}{[#1]}}

\setlength{\parskip}{0.4em}

\title{天色の四季 \\ Weather of the four seasons}
\author{Original Source: 暁records --- Translated by: Othi}
\begin{document}

\maketitle

\tableofcontents
\section{Origin}
\noindent Original game: 東方天空璋 ~ Hidden Star in Four Seasons. \\
Original title: 魔法の笠地蔵 \\
Theme: Stage 4 BOSS --- Yatadera Narumi's (矢田寺成美) theme \\
Original title: 一対の神獣 \\
Theme: Stage 3 BOSS --- Komano Aunn's (高麗野 あうん) theme \\
Original title: 秘匿されたフォーシーズンズ \\
Theme: Stage 6 BOSS --- Matara Okina's (摩多羅 隠岐奈) theme

\section{Summary}
Okina Matara being the perpetrator to Gensokyo's seasons being out of order everywhere. The song portrays a milder storytelling of the cast in Hidden Star in Four Seasons.

\pagebreak
\section{Lyric}
\subsection{Verse \#1}
\furigana{旅}{たび}\furigana{立}{たつ}ちを\furigana{告}{し}げる\furigana{日差}{ひざ}し \\
The sunlight marks the start of our journey

\furigana{映}{は}える\furigana{草}{くさ}\furigana{萌}{も}えのコントラスト \\
The contrast of the shining grass buds

\furigana{目覚}{めざ}め\furigana{際}{きわ}の\furigana{鮮}{あざ}やかな\furigana{夢}{ゆめ}は \\
Being on the verge of waking up from this bright dream

まるで\furigana{走馬灯}{そうまとう}のように \\
Really feels as if they were flashing before my eyes

\furigana{天色}{あまいろ}の\furigana{四季}{しき}を\furigana{走}{はし}り\furigana{抜}{ぬ}けてく \\
The four seasons' weather come and go

\furigana{幻想}{げんそう}\furigana{機関車}{きかんしゃ}の\furigana{汽笛}{きてき} \\
The phantasy train's whistle \tlnotelink{steam}

\subsection{Chorus \#1}
\furigana{焼}{や}き\furigana{付}{つ}けて もう\furigana{巡}{めぐ}らない\furigana{今}{いま}を \\
Burn the momentary tranquil season in into your memories

\furigana{同}{おな}じ\furigana{季節}{きせつ}は\furigana{二度}{にど}と\furigana{来}{こ}ない \\
The same season that will never returns

\furigana{重}{かさ}なる\furigana{瞬間}{しゅんかん} \furigana{連}{つら}なる\furigana{時空}{じくう} \\
These piled up moments that connects space and time

その\furigana{狭間}{はざま}に\furigana{生}{い}きて \\
Savour them and even though

\furigana{忘}{わす}れても、\furigana{失}{うしな}うわけじゃない \\
You might forget them, but you will not truly lose it

\furigana{天色}{あまいろ}の\furigana{四季}{しき}に\furigana{虹}{にじ}がかかる \\
The color of the four seasons comes into view

\subsection{Verse \#2}
うだる\furigana{暑}{あつ}さの\furigana{遠}{とお}い\furigana{日々}{ひび}は \\
As for the boiling heat in the coming days

\furigana{永遠}{えいえん}そのものだったのに \\
Even though it was forever

あのころ、\furigana{特別}{とくべつ}だったものは \\
Those days were special

ぜんぶ\furigana{色褪}{いろあ}せてしまった \\
When everything has faded away

\furigana{天色}{あまいろ}の\furigana{四季}{しき}を\furigana{走}{はし}り\furigana{抜}{ぬ}けてく \\
The colors of the four seasons come and go

\furigana{幻想}{げんそう}\furigana{機関車}{きかんしゃ}に\furigana{乗}{の}り \\
Take a ride on this fantasy train \tlnotelink{steam}

\subsection{Chorus \#2}
\furigana{焼}{や}き\furigana{付}{つ}けて もう\furigana{巡}{めぐ}らない\furigana{今}{いま}を \\
Burn the momentary tranquil season in into your memories

\furigana{同}{おな}じ\furigana{季節}{きせつ}は\furigana{二度}{にど}と\furigana{来}{こ}ない \\
The same season that will never returns

\furigana{重}{かさ}なる\furigana{瞬間}{しゅんかん} \furigana{連}{つら}なる\furigana{時空}{じくう} \\
These piled up moments that connects space and time

その\furigana{狭間}{はざま}に\furigana{生}{い}きて \\
Savour them and

いつのまにか\furigana{遠}{とお}くに\furigana{来}{き}たね \\
before you notice, you've come so far

\furigana{天色}{あまいろ}の\furigana{四季}{しき}に\furigana{虹}{にじ}がかかる \\
The color of the four seasons comes into view

\subsection{Verse \#3}
\furigana{遠}{おと}ざかる\furigana{線路}{せんろ}に\furigana{残}{のこ}した\furigana{記憶}{きおく} \\
The memory that was left behind on the fading railway

\furigana{笑}{わら}って\furigana{手}{て}を\furigana{振}{ふ}っていたよ \\
Your laughter as you were waving your hand

そして \furigana{新}{あたら}しい\furigana{景色}{けしき}と\furigana{出逢}{であ}う \\
As you encounter a new scenery

やがて \furigana{汽車}{きしゃ}が\furigana{止}{と}まるときまで \\
Eventually, until the imaginary train halts its track

ひとつひとつを\furigana{目}{め}に\furigana{焼}{や}きつけ
Remember every single one of them clearly

ほら、\furigana{今日}{きょう}も\furigana{初}{はじ}めての\furigana{季節}{きせつ}
Look, today is already the start of a new season

\furigana{天色}{あまいろ}の\furigana{四季}{しき}に\furigana{虹}{にじ}がかかる \\
The color of the four seasons comes into view

\section{Translator's note}
\tlnoteref{steam}: the use of steam train seemed to be a metaphor to how it keeps moving and every stops that you might take would be different from each other --- just like the constantly changing and vastly different seasons throughout Gensokyo
\end{document}
