%!TEX TS-program = xelatex
% !TeX program = xelatex
\documentclass{article}
\usepackage[margin=1in]{geometry}
\usepackage{xeCJK}
\usepackage{ruby}
\setCJKmainfont{NotoSansJP-Light.otf}
\usepackage{tikz}
\usepackage{comment}
\usepackage{indentfirst}
\usepackage[linktoc=all]{hyperref}
% characters with big height and depth
% to give different boxes the same vertical size
\newcommand{\vsizecorrectorhira}{\vphantom{もりぼゃ}}% epkyouka, HanaMinA
\newcommand{\vsizecorrectorkanji}{\vphantom{$\vert$}}

% commands for setting furigana below kanji
\newcommand{\furiganaBelow}[2]{% #1: kanji, #2: furigana
    %\unskip
    \begin{tikzpicture}[baseline=(kanji.base)]
        \node(kanji)[
            inner sep=0,
        ] {
            \vsizecorrectorkanji%
            #1%
        };
        \node(furigana)[
            below of=kanji,
            node distance=1em-2pt,
            inner sep=0,
        ] {%
            \tiny%
            \vsizecorrectorhira%
            #2%
        };
    \end{tikzpicture}%
    %\ignorespaces
}

\newcommand{\furiganaAboveBelow}[3]{% #1: kanji, #2: furigana above, #3: furigana below
    %\unskip
    \begin{tikzpicture}[baseline=(kanji.base)]
        \node(kanji)[
            inner sep=0,
        ] {
            \vsizecorrectorkanji%
            #1%
        };
        \node(furigana-above)[
            above of=kanji,
            node distance=1em,
            inner sep=0,
        ] {%
            \tiny%
            \vsizecorrectorhira%
            #2%
        };
        \node(furigana-below)[
            below of=kanji,
            node distance=1em-2pt,
            inner sep=0,
        ] {%
            \tiny%
            \vsizecorrectorhira%
            #3%
        };
    \end{tikzpicture}%
    %\ignorespaces
}

\newcommand{\furigana}[2]{% #1: kanji, #2: furigana
    %\unskip
    \begin{tikzpicture}[baseline=(kanji.base)]
        \node(kanji)[
            inner sep=0,
        ] {
            \vsizecorrectorkanji%
            #1%
        };
        \node(furigana)[
            above of=kanji,
            node distance=1em-0pt,
            inner sep=0,
        ] {%
            \scriptsize%
            \vsizecorrectorhira%
            #2%
        };
    \end{tikzpicture}%
    %\ignorespaces
}

\newcommand{\tlnotelink}[1]{\hyperlink{#1tonote}{[#1]}\hypertarget{#1}{}}
\newcommand{\tlnoteref}[1]{\hypertarget{#1tonote}{}\hyperlink{#1}{[#1]}}

\setlength{\parskip}{0.4em}

\title{幽鬼傀儡、あるいはその愛と再生 \\ Spirit puppet, or perhaps love and rebirth}
\author{Original Source: 暁records --- Translated by: Othi}
\begin{document}

\maketitle

\tableofcontents
\section{Origin}
\noindent Original game: 東方神霊廟 ~ Ten Desires

\noindent Original title: 古きユアンシェン \\
Theme: Stage 4 BOSS --- Kaku Seiga's (霍 青娥) theme \\
Original title: リジッドパラダイス \\
Theme: Stage 3 BOSS --- Miyako Yoshika's (宮古 芳香) theme

\section{Summary}
The song is sung from \href{https://en.touhouwiki.net/wiki/Yoshika_Miyako}{Miyako Yoshiva's} perspective, describing her unending circle of life and death bridged by resurrection as a \href{https://en.wikipedia.org/wiki/Jiangshi}{Jiangshi} revived by \href{https://en.touhouwiki.net/wiki/Seiga_Kaku}{Seiga Kaku} as her experiment for \href{https://en.wikipedia.org/wiki/Taoism}{Taoism} and necromancy. It also portrays the twisted master-servant relationship between the two individuals.

\pagebreak

\section{Lyric}

\subsection{Intro}

(\furigana{眠}{ねむ}りその\furigana{時}{とき} \furigana{蘇る}{よみがえ}よ) \\
I'm resurrected as I fall to sleep

\furigana{雷鳴}{らいめい}に\furigana{開}{ひら}く\furigana{眼}{め} \furigana{泥濘}{ぬかる\(む\)}んだ\furigana{土}{つち}、\furigana{掴}{つか}む\furigana{手}{て} \\
My eyes opened to the thunderclap as you catch my hands on the muddy ground


\furigana{長}{なが}い\furigana{爪}{つめ}が\furigana{抉}{えぐ}り \furigana{再}{ふたた}を\furigana{肺}{はい}を\furigana{満}{み}たす\furigana{命}{いのち} \\
These long nails gouging out this life as it fills my lung once again

\subsection{Verse \#1}

なにひとつ、\furigana{残}{のこ}せはしない 「\furigana{思}{おも}い\furigana{出}{で}、\furigana{愛}{あい}に\furigana{名声}{めいせい}」 \\
I won't leave a single thing behind 「Memories are love's signature」

なにひとつ、\furigana{持}{も}ち\furigana{出}{だ}せない 「\furigana{死}{し}はすべてを\furigana{奪}{うば}う」 \\
I can't bring a single thing back  「Death taketh everything away」

ただひとつ、\furigana{確}{たし}かなもの   「さあ、私とおいで」 \\
However there's still one thing for certain 「Now come to me」

私を\furigana{突}{つ}き\furigana{動}{うご}かす\furigana{衝動}{しょうどう}    「\furigana{貴女}{あなた}は\furigana{愛}{いと}しい\furigana{傀儡}{かいらい}」 \\
Impelling urges to stab me 「You are my lovely puppet」

\subsection{Chorus \#1}

あの\furigana{日}{ひ}、いつかの\furigana{日}{ひ}に、私は\furigana{死}{し}に。そして\furigana{今}{いま}\furigana{蘇}{よみがえ}る \\
Some day, one day I'll be dead. Then soon i'll return to life.

\furigana{自}{みずか}ら\furigana{目指}{めざ}す\furigana{当}{あ}てもなく、\furigana{詩}{うた}も\furigana{忘}{わす}れて。 \\
Your goal is so aimless, just forget your poem already
\tlnotelink{1}

\furigana{何度}{なんど}でも\furigana{死}{し}ぬために。 \\
Please die for me over and over again

\subsection{Breakdown \#1}

(あぁ、\furigana{可愛}{わかい}いものこの\furigana{傀儡}{かいらい}) (私のためにキスをして\furigana{何度}{なんど}でも\furigana{死}{し}ぬがいい) \\
(Aaa, this cute thing of a puppet) (You can kiss me and die for me as many times as you like)

(ずっと私だけのモノにしてあげる) \\
I'll have you do everything for me for eternity
\tlnotelink{2}

(バラバラに\furigana{壊}{こわ}れても \furigana{無残}{むざん}に\furigana{引}{ひ}きちぎられても) (\furigana{彼}{かれ}も\furigana{臓物}{ぞうもつ}も\furigana{全}{すべ}ては私の\furigana{物}{もの}) \\
(Even if you break me to pieces, even if you tear me off savagely) (All of her entrails will still be mine)

(私は\furigana{認}{みと}めない) (\furigana{誰}{だれ}にも\furigana{渡}{わた}さない) \\
(I won't acknowledge it) (I won't hand her over to anyone)

「\furigana{何}{なに}も\furigana{感}{かん}じなくていいの \furigana{捥}{もぐ}がれた\furigana{華}{はな}	私のそばにおいでなさい」 \\
「You don't have to feel anything, O whithered flower, please come by my side」

「\furigana{何}{なに}も\furigana{思}{おも}い\furigana{出}{だ}さなくていいのよ	\furigana{貴女}{あなた}ってば、\furigana{腐}{くさ}って\furigana{尚更}{なおさら}\furigana{美}{うつく}しいわね」 \\
You don't have to think of anything anymore, aren't you even more beautiful when you rot away?

\furigana{紅葉}{こうよう}\furigana{降}{ふ}る\furigana{秋空}{あきぞら}\furigana{見上}{みあ}げては \\
As I looked up at the falling maple in the autumn sky,
\tlnotelink{3}

\furigana{誰}{だれ}か、\furigana{誰}{だれか}かに\furigana{呼}{よ}ばれているような\furigana{気}{き}がして \\
came the feeling as if someone was calling out to me

きっとここに\furigana{眠}{ねむ}る私のために、ひとり\furigana{泣}{な}いてくれた \\
who was surely crying when I was laid to rest here.

\furigana{優}{やさ}しい人の\furigana{声}{こえ}なんだろう \\
I wonder whose that person's voice was

\subsection{Chorus \#2}

あの\furigana{日}{ひ}、いつかの\furigana{日}{ひ}に、私は\furigana{死}{し}に。そして\furigana{今}{いま}\furigana{蘇}{よみがえ}る \\
Some day, one day I'll be dead. Then soon I'll return to life.

\furigana{自}{みずか}ら\furigana{目指}{めざ}す\furigana{当}{あ}てもなく、\furigana{詩}{うた}も\furigana{忘}{わす}れて。 \\
Your goal is so aimless, just forget your poem already

\furigana{何度}{なんど}でも\furigana{死}{し}ぬために。 \\
Please die for me over and over again

\subsection{Verse \#2}

\furigana{死}{し}に\furigana{見初}{みぞ}められた\furigana{傀儡}{かいらい}の\furigana{詩}{うた} \furigana{逃}{のが}れられない\furigana{鳥籠}{とりかご} \\
A poem about the puppet that fell in love at first sight with death. Such is the unescapable birdcage

\furigana{夢}{ゆめ}\furigana{見}{み}て、その\furigana{眠}{ねむ}り\furigana{浅}{あさ}く \furigana{転寝}{ごろね}のよう \\
Her dreaming seemingly light as a nap
\tlnotelink{4}

\furigana{引}{ひ}き\furigana{伸}{の}ばされる\furigana{魂}{たみしい} \\
The ever \furiganaBelow{expanding}{resurrecting} spirit

\section{Translation notes}

\tlnoteref{1}: This is the poem usually cited by Miyako. Yoshika Miyako's charater design in the Touhou universre is based on \href{https://ja.wikipedia.org/wiki/%E9%83%BD%E8%89%AF%E9%A6%99}{Yoshika no Miyako (都良香)}, a Japanese scholar and poet. \\
    Furthermore, various fan-made works \href{https://en.wikipedia.org/wiki/D%C5%8Djin}{同人} including official comic \href{https://www.ichijinsha.co.jp/special/toho_ibarakasen/}{Wild and Horned Hermit} showed in the first chapters Kasen's interest in Miyako's poems, hinting that citing poems is one of Miyako's main activities during the time that she was still alive.

\tlnoteref{2}: This section portrays Seiga's possessive mental state and her obsession towards Miyako and Taoism. Even though she seems to be treating Miyako as a experiment for her conductions of Taoism and a guard, she was shown deep affection towards Miyako and kept reviving her back to life diligently.

\tlnoteref{3}: Signs of Seiga's affectionate relationship with Miyako specifically, even though Seiga is a hermit that only wants to advance her own goal and treat everyone else equally lowly and would rather manipulate them for her own benefits instead. Miyako who is supposedly her servant is the only exception of this.

\tlnoteref{4}: A metaphor to Miyako's resurrection process that is seemingly as light as a short nap as life fills her lung again shortly after.

Personal translator's note: Othi here, this is my first attempt at translating a song lyric. This song specifically was my personal favorite Touhou Project arrangement and it gave me the motivation needed to want to know more about the song, the backstory of included characters and their relationships, and furthermore the push I needed to pick up a new language to learn and to translate them in hope to bring this piece of literature to a wider audience. It also shows how beautiful and intricate the Touhou universe can be.

To read until the very end, I thank you and appreciate your time that you've spent and I hope you can find enjoyment in reading my future works.
\end{document}
