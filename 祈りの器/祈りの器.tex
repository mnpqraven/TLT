%!TEX TS-program = xelatex
% !TeX program = xelatex
\documentclass{article}
\usepackage[margin=1in]{geometry}
\usepackage{xeCJK}
\usepackage{ruby}
\setCJKmainfont{NotoSansJP-Light.otf}
\usepackage{tikz}
\usepackage{comment}
\usepackage{indentfirst}
\usepackage[linktoc=all]{hyperref}
% characters with big height and depth
% to give different boxes the same vertical size
\newcommand{\vsizecorrectorhira}{\vphantom{もりぼゃ}}% epkyouka, HanaMinA
\newcommand{\vsizecorrectorkanji}{\vphantom{$\vert$}}

% commands for setting furigana below kanji
\newcommand{\furiganaBelow}[2]{% #1: kanji, #2: furigana
    %\unskip
    \begin{tikzpicture}[baseline=(kanji.base)]
        \node(kanji)[
            inner sep=0,
        ] {
            \vsizecorrectorkanji%
            #1%
        };
        \node(furigana)[
            below of=kanji,
            node distance=1em-2pt,
            inner sep=0,
        ] {%
            \tiny%
            \vsizecorrectorhira%
            #2%
        };
    \end{tikzpicture}%
    %\ignorespaces
}

\newcommand{\furiganaAboveBelow}[3]{% #1: kanji, #2: furigana above, #3: furigana below
    %\unskip
    \begin{tikzpicture}[baseline=(kanji.base)]
        \node(kanji)[
            inner sep=0,
        ] {
            \vsizecorrectorkanji%
            #1%
        };
        \node(furigana-above)[
            above of=kanji,
            node distance=1em,
            inner sep=0,
        ] {%
            \tiny%
            \vsizecorrectorhira%
            #2%
        };
        \node(furigana-below)[
            below of=kanji,
            node distance=1em-2pt,
            inner sep=0,
        ] {%
            \tiny%
            \vsizecorrectorhira%
            #3%
        };
    \end{tikzpicture}%
    %\ignorespaces
}

\newcommand{\furigana}[2]{% #1: kanji, #2: furigana
    %\unskip
    \begin{tikzpicture}[baseline=(kanji.base)]
        \node(kanji)[
            inner sep=0,
        ] {
            \vsizecorrectorkanji%
            #1%
        };
        \node(furigana)[
            above of=kanji,
            node distance=1em-0pt,
            inner sep=0,
        ] {%
            \scriptsize%
            \vsizecorrectorhira%
            #2%
        };
    \end{tikzpicture}%
    %\ignorespaces
}

\setlength{\parskip}{0.4em}

\title{祈りの器 \\ Vessel for prayers}
\author{Original Source: 暁records --- Translated by: Othi}
\begin{document}

\maketitle

\tableofcontents
\section{Origin}
\noindent Original game: 東方鬼形獣 ~ Wily Beast and Weakest Creature \\

\noindent Original title: エレクトリックヘリテージ ~ Electric Heritage \\
Theme: Stage 6 theme

\noindent Original title: 偶像に世界を委ねて ~ Idoratrize World \\
Theme: Stage 6 Boss --- Haniyasushin Keiki's (埴安神 袿姫) theme \\

\section{Summary}

\pagebreak

\section{Lyric}

\begin{verbatim}
Vessel can be inferred as the idols from mayuki's idol army, which is powered by faith directed to them
\end{verbatim}
\subsection{Verse \#1}
\furigana{空}{から}っぽの\furigana{瞳}{ひとみ}に\furigana{浮}{う}かんでは\furigana{消}{き}えた \\
It kept coming in and out of sight of these empty pupils

\furigana{交}{ま}わらない\furigana{誰}{だれ}かの\furigana{影}{かげ} \\
I don't come across any shadow

\furigana{笑}{わら}っている どこかで \\
As i'm laughing

\furigana{遠}{とお}のいていく\furigana{音影}{おんがく} \\
There's music from somewhere faraway


ここに\furigana{置}{お}き\furigana{去}{さ}りの\furigana{僕}{ぼく}らは \\
We are left behind here


\subsection{Chorus \#1}
どうか、この\furigana{祈}{いの}りが\furigana{届}{とど}くのなら \\
If this prayer can somehow reach you


まだ\furigana{神}{かみ}がいるなら \\
If god still exists

なにか\furigana{信}{しん}じられる\furigana{希望}{きぼう}を \furigana{光}{ひかり}を \\
Something that we can believe in, perhaps a hope, perhaps some light

ああ、\furigana{空}{そら}に\furigana{星}{ほし}は\furigana{絶}{た}えて \\
There, died out stars in the sky


\furigana{花}{はな}は\furigana{色}{いろ}を\furigana{失}{うしな}ってしまうから \\
Flowers sadly void of color

\furigana{奪}{うば}えない\furigana{祈}{いの}りの\furigana{器}{うしわ}を \\
The vessel for prayers that can't be taken away

\subsection{Verse \#2}
\furigana{悲}{かな}しみを\furigana{量}{はか}って\furigana{比}{くら}べて\furigana{競}{きそ}って \\
Weigh your grief, appraise it, contend it


\furigana{例}{たと}え「\furigana{幸福}{こうふく}だ」って\furigana{言}{い}われたって \\
If you have said `This is a blessing'


\furigana{浮}{う}かんだ\furigana{血}{ち}は\furigana{癒}{い}えない \\
This spewed blood won't return


\furigana{傷口}{きずぐち}はふさがらない \\
These wounds won't be closed

\furigana{頬}{ほお}を\furigana{流}{なが}れ\furigana{渇}{かわ}いた\furigana{涙}{なみだ} \\
We craved for these streams of tears running down your cheeks

\subsection{Chorus \#2}
どうか、この\furigana{祈}{いの}りが\furigana{届}{とど}くのなら \\
If this prayer can somehow reach you

まだ\furigana{神}{かみ}がいるなら
If god still exists

なにか\furigana{信}{しん}じられる\furigana{希望}{きぼう}を \furigana{光}{ひかり}を \\
Something that we can believe in, perhaps a hope, perhaps some light

ああ、\furigana{空}{そら}に\furigana{星}{ほし}は\furigana{絶}{た}えて \\
There, died out stars in the sky

\furigana{花}{はな}は\furigana{色}{いろ}を\furigana{失}{うしな}ってしまうから \\
Flowers sadly void of color

\furigana{奪}{うば}えない\furigana{祈}{いの}りの\furigana{器}{うしわ}を \\
The vessel for prayers that can't be taken away

\subsection{Verse \#3}
\furigana{神}{きま}\furigana{様}{さま}、この\furigana{祈}{いの}りが この\furigana{想}{おも}いが \\
O God, these prayers, these thoughts


まだ\furigana{聞}{き}こえるのなら \\
If those can still be heard

\furigana{力}{ちから}なき\furigana{僕}{ぶく}らに\furigana{最後}{さいご}の\furigana{救}{すく}いを \\
We'll loudly cry out for your final salvation

ああ、\furigana{空}{そら}に\furigana{星}{ほし}\furigana{輝}{かがや}き \\
There, radiant stars in the sky

\furigana{花}{はな}の\furigana{色}{いろ}は\furigana{数}{かず}え\furigana{切}{き}れないはず \\
The flower's countless amount of color

\furigana{信}{しん}じられるものがあるなら \\
If such believing creature exists

\furigana{奪}{うば}えない\furigana{祈}{いの}りの\furigana{器}{うつわ}よ \\
The vessel for prayers will never be taken away
\end{document}
